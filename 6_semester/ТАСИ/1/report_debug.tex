% Page 
% = setup 
\documentclass[a4paper]{article}
\usepackage[14pt]{extsizes}
\usepackage[T2A]{fontenc}
\usepackage{ dsfont }
\usepackage[utf8]{inputenc}
\usepackage[russian]{babel}
\usepackage{csquotes}
% = margins
\usepackage[left=3cm,right=1.5cm,top=2cm,bottom=2cm]{geometry}
% = indents
\usepackage{indentfirst}
\setlength{\parindent}{1.25cm}
% = spacing
\usepackage{setspace}
\onehalfspacing

% Bibliography
\usepackage[square,sort,comma,numbers]{natbib}
\setlength{\bibhang}{0pt}
\makeatletter
\renewcommand{\@biblabel}[1]{}
\makeatother
\makeatletter
\newcounter{bibcounter}
\@ifundefined{CSLReferences}{
  \newenvironment{CSLReferences}[2]
   {\list{}
    {\setlength{\leftmargin}{\bibindent}
     \setlength{\itemindent}{-\leftmargin}
     \setlength{\itemsep}{\bibsep}
     \setlength{\parsep}{\z@}}}
   {\endlist}
}{}
\let\oldbibitem\bibitem
\renewcommand{\bibitem}[2][]{\stepcounter{bibcounter}\oldbibitem{#2}}
\makeatother

\newcommand{\citeproctext}[1]{#1}

% Fonts
% = family
\usepackage{fontspec}
\setmainfont{Times New Roman}
\setmonofont{Courier New}

% Text
% = parindent
\setlength{\parindent}{1.25cm}
% = strikethrough
\usepackage{ulem}
\let\st\sout
% = italic
\renewcommand{\emph}[1]{\textit{#1}}

% Title 
% = depth
\setcounter{secnumdepth}{4}
\setcounter{tocdepth}{4}
% = space after title
\usepackage{titlesec}
\titleformat{\section}
  {\normalfont\normalsize\bfseries\sloppy}{\hspace{1.25cm}\thesection}{1em}{}
\titleformat{\subsection}
  {\normalfont\normalsize\bfseries\sloppy}{\hspace{1.25cm}\thesubsection}{1em}{}
\titleformat{\subsubsection}
  {\normalfont\normalsize\bfseries\sloppy}{\hspace{1.25cm}\thesubsubsection}{1em}{}
\titleformat{\paragraph}
  {\normalfont\normalsize\bfseries\sloppy}{\hspace{1.25cm}\theparagraph}{1em}{}
% = Adjust spacing
\titlespacing*{\section}{0pt}{0.3\baselineskip}{0.3\baselineskip}
\titlespacing*{\subsection}{0pt}{0.3\baselineskip}{0.3\baselineskip}
\titlespacing*{\subsubsection}{0pt}{0.3\baselineskip}{0.3\baselineskip}
\titlespacing*{\paragraph}{0pt}{0.3\baselineskip}{0.3\baselineskip}
% = breaks
\usepackage{titlesec}
\newcommand{\sectionbreak}{\clearpage}
% = special title
\newcommand{\centertitle}[1]{
  \titleformat*{\section}{\centering\normalsize\bfseries}
  \section*{#1}
  \titleformat*{\section}{\normalfont\normalsize\bfseries\sloppy}
  \addcontentsline{toc}{section}{#1}
}

% List
% = indents
\usepackage{enumitem}
\setlist[itemize,enumerate,1]{label=\textbullet, topsep=0.25pt, partopsep=0.25pt, parsep=0pt, itemsep=0pt, labelindent=\parindent, align=left, leftmargin=*}
\setlist[itemize,enumerate,2]{label=\textbullet, topsep=0.25pt, partopsep=0.25pt, parsep=0pt, itemsep=0pt, labelindent=\parindent, align=left, leftmargin=*}
\setlist[itemize,enumerate,3]{label=\textbullet, topsep=0.25pt, partopsep=0.25pt, parsep=0pt, itemsep=0pt, labelindent=\parindent, align=left, leftmargin=*}
\setlist[itemize,enumerate,4]{label=\textbullet, topsep=0.25pt, partopsep=0.25pt, parsep=0pt, itemsep=0pt, labelindent=\parindent, align=left, leftmargin=*}
% = tightlist fix
\def\tightlist{}

% Сode
\usepackage{xcolor}
\definecolor{lightgray}{rgb}{0.95, 0.95, 0.95}
\usepackage{listings}
\lstset{
  basicstyle=\small,
  numbers=left,                    
  stepnumber=1,                   
  numbersep=10pt,                  
  tabsize=2,                      
  showspaces=false,                
  showstringspaces=false,
  breaklines=true,                
  escapeinside={\%*}{*)},         
  numbersep=15pt,                 
  xleftmargin=25pt,              
  backgroundcolor=\color{lightgray}
}
% = inline code
\usepackage{fontspec}
\newfontfamily{\codefont}[SizeFeatures={Size=14}]{Courier New}
\newcommand{\passthrough}[1]{{\codefont #1}}
% = fix russian sybmols
\makeatletter % see https://tex.stackexchange.com/a/320345
\lst@InputCatcodes
\def\lst@DefEC{%
 \lst@CCECUse \lst@ProcessLetter
  ^^80^^81^^82^^83^^84^^85^^86^^87^^88^^89^^8a^^8b^^8c^^8d^^8e^^8f%
  ^^90^^91^^92^^93^^94^^95^^96^^97^^98^^99^^9a^^9b^^9c^^9d^^9e^^9f%
  ^^a0^^a1^^a2^^a3^^a4^^a5^^a6^^a7^^a8^^a9^^aa^^ab^^ac^^ad^^ae^^af%
  ^^b0^^b1^^b2^^b3^^b4^^b5^^b6^^b7^^b8^^b9^^ba^^bb^^bc^^bd^^be^^bf%
  ^^c0^^c1^^c2^^c3^^c4^^c5^^c6^^c7^^c8^^c9^^ca^^cb^^cc^^cd^^ce^^cf%
  ^^d0^^d1^^d2^^d3^^d4^^d5^^d6^^d7^^d8^^d9^^da^^db^^dc^^dd^^de^^df%
  ^^e0^^e1^^e2^^e3^^e4^^e5^^e6^^e7^^e8^^e9^^ea^^eb^^ec^^ed^^ee^^ef%
  ^^f0^^f1^^f2^^f3^^f4^^f5^^f6^^f7^^f8^^f9^^fa^^fb^^fc^^fd^^fe^^ff%
  ^^^^20ac^^^^0153^^^^0152%
  % Basic Cyrillic alphabet coverage
  ^^^^0410^^^^0411^^^^0412^^^^0413^^^^0414^^^^0415^^^^0416^^^^0417%
  ^^^^0418^^^^0419^^^^041a^^^^041b^^^^041c^^^^041d^^^^041e^^^^041f%
  ^^^^0420^^^^0421^^^^0422^^^^0423^^^^0424^^^^0425^^^^0426^^^^0427%
  ^^^^0428^^^^0429^^^^042a^^^^042b^^^^042c^^^^042d^^^^042e^^^^042f%
  ^^^^0430^^^^0431^^^^0432^^^^0433^^^^0434^^^^0435^^^^0436^^^^0437%
  ^^^^0438^^^^0439^^^^043a^^^^043b^^^^043c^^^^043d^^^^043e^^^^043f%
  ^^^^0440^^^^0441^^^^0442^^^^0443^^^^0444^^^^0445^^^^0446^^^^0447%
  ^^^^0448^^^^0449^^^^044a^^^^044b^^^^044c^^^^044d^^^^044e^^^^044f%
  ^^^^0401^^^^0451%
  %%%
  ^^00}
\lst@RestoreCatcodes
\makeatother

% Hyperlinks
\usepackage[hidelinks]{hyperref}
\let\oldhref\href
% = URI
\usepackage{seqsplit}
\renewcommand{\href}[2]{\oldhref{#1}{#2} (URI - \url{#1})}

% Images
\usepackage{graphicx}
\usepackage{adjustbox}
\usepackage{caption}
\usepackage{float}

\DeclareCaptionLabelSeparator{mysep}{ - }
\captionsetup[figure]{name=Рисунок, labelsep=mysep}
\renewcommand{\thefigure}{\arabic{figure}}

\newcommand{\image}[3]{
    % fix floating
    \begin{figure}[H]
        \centering
        \begin{adjustbox}{width=#3\textwidth,center}
            \includegraphics[keepaspectratio]{\detokenize{#1}}
        \end{adjustbox}
        \caption{\detokenize{#2}}
        \label{fig:#2}
    \end{figure}
}

% Table
\usepackage{longtable}
\usepackage{booktabs}
\usepackage{array}
% = simple creation
\captionsetup[table]{
  name=Таблица, 
  labelsep=mysep,
  justification=raggedleft,
  singlelinecheck=false
}
% = csv read
\usepackage{pgfplotstable}
\newcommand{\sucsvtable}[2]{
\begin{table}[h]
\caption{#2}
\centering
\renewcommand{\arraystretch}{1.5} 
\pgfplotstabletypeset[
  col sep=semicolon,
  string type,
  every head row/.style={before row=\hline,after row=\hline},
  every column/.style={column type={|c}},
  every last column/.style={column type={|c|}},
  after row=\hline
]{#1}
\label{tab:#2}
\end{table}
}
% = label 
\let\oldcaption\caption
\renewcommand{\caption}[1]{\oldcaption{#1}\label{tab:#1}}

\newcommand{\sutable}[2]{
\begin{table}[h!]
    \centering
    \caption{#1}
    \label{table:#1}
    \bgroup
      \def\arraystretch{1.3}
      {#2}
    \egroup
  \end{table}
}

% Equation
\usepackage{amsmath}
\usepackage{amssymb}
\usepackage{etoolbox}
% = line margin 
% TODO: GOST
\BeforeBeginEnvironment{equation}{\vspace{-0.6\baselineskip}}
\BeforeBeginEnvironment{equation*}{\vspace{-0.6\baselineskip}}

% Table of content
\usepackage{titlesec}
\usepackage{tocloft}
\renewcommand{\cfttoctitlefont}{\hfill\normalsize\bfseries\MakeUppercase}
\renewcommand{\cftaftertoctitle}{\hfill\mbox{}}
\renewcommand{\cftsecleader}{\cftdotfill{\cftdotsep}}

% Links numeration
\usepackage{chngcntr}
\counterwithin{figure}{section}
\counterwithin{table}{section}

\usepackage{grffile}
\usepackage{pdfpages}
\begin{document}
\sloppy
\tableofcontents
****--- csl:
/Users/razrab-ytka/Documents/Projects/suaidoc/suaidoc/templates/gost2008.csl
suaidocintropath:
\detokenize{/var/folders/mr/twkc4jg14f3cl5vr1lh0kgl80000gn/T/tmpxrm7edwb/intro.pdf}
department: 42 teacher: Суетина Т. А. teacher\_title: Доцент theme:
Энтропийные алгоритмы сжатия информации variant: 5 number: 1 discipline:
Техника аудиовизуальных средств информации group: 4128 student: Воробьев
В. А. ---

\section{Введение}\label{ux432ux432ux435ux434ux435ux43dux438ux435}

\subsection{Цель лабораторной
работы}\label{ux446ux435ux43bux44c-ux43bux430ux431ux43eux440ux430ux442ux43eux440ux43dux43eux439-ux440ux430ux431ux43eux442ux44b}

Освоить алгоритмы для сжатия информации.

\subsection{Задание}\label{ux437ux430ux434ux430ux43dux438ux435}

Выполнить сжатие текста 4 способами:

\begin{itemize}
\tightlist
\item
  Метод Хаффмана;
\item
  Метод Шенона-Фано;
\item
  Арифметическим кодированием;
\item
  Алгоритмом LZW.
\end{itemize}

Для каждого метода рассчитать коэффициент сжатия текста.

\textbf{Вариант 5:} ШОРОХ ОТ ДУБКА КАК БУДТО ХОРОШ

\section{Выполнение
работы}\label{ux432ux44bux43fux43eux43bux43dux435ux43dux438ux435-ux440ux430ux431ux43eux442ux44b}

\subsection{Теоретические
сведения}\label{ux442ux435ux43eux440ux435ux442ux438ux447ux435ux441ux43aux438ux435-ux441ux432ux435ux434ux435ux43dux438ux44f}

\begin{equation}\begin{gathered}
K = \frac{V_{\mathit{вх}}}{V_{\mathit{вых}}},
\end{gathered}\end{equation}

где K - степень сжатия.

\subsection{Анализ исходного
текста}\label{ux430ux43dux430ux43bux438ux437-ux438ux441ux445ux43eux434ux43dux43eux433ux43e-ux442ux435ux43aux441ux442ux430}

Для начала проанализируем текст.

\begin{longtable}[]{@{}
  >{\raggedright\arraybackslash}p{(\columnwidth - 22\tabcolsep) * \real{0.1463}}
  >{\centering\arraybackslash}p{(\columnwidth - 22\tabcolsep) * \real{0.0732}}
  >{\centering\arraybackslash}p{(\columnwidth - 22\tabcolsep) * \real{0.0732}}
  >{\centering\arraybackslash}p{(\columnwidth - 22\tabcolsep) * \real{0.0732}}
  >{\centering\arraybackslash}p{(\columnwidth - 22\tabcolsep) * \real{0.0732}}
  >{\centering\arraybackslash}p{(\columnwidth - 22\tabcolsep) * \real{0.1220}}
  >{\centering\arraybackslash}p{(\columnwidth - 22\tabcolsep) * \real{0.0732}}
  >{\centering\arraybackslash}p{(\columnwidth - 22\tabcolsep) * \real{0.0732}}
  >{\centering\arraybackslash}p{(\columnwidth - 22\tabcolsep) * \real{0.0732}}
  >{\centering\arraybackslash}p{(\columnwidth - 22\tabcolsep) * \real{0.0732}}
  >{\centering\arraybackslash}p{(\columnwidth - 22\tabcolsep) * \real{0.0732}}
  >{\centering\arraybackslash}p{(\columnwidth - 22\tabcolsep) * \real{0.0732}}@{}}
\caption{Количество вхождений символов.}\tabularnewline
\toprule\noalign{}
\begin{minipage}[b]{\linewidth}\raggedright
Буква
\end{minipage} & \begin{minipage}[b]{\linewidth}\centering
Ш
\end{minipage} & \begin{minipage}[b]{\linewidth}\centering
О
\end{minipage} & \begin{minipage}[b]{\linewidth}\centering
Р
\end{minipage} & \begin{minipage}[b]{\linewidth}\centering
Х
\end{minipage} & \begin{minipage}[b]{\linewidth}\centering
space
\end{minipage} & \begin{minipage}[b]{\linewidth}\centering
Д
\end{minipage} & \begin{minipage}[b]{\linewidth}\centering
У
\end{minipage} & \begin{minipage}[b]{\linewidth}\centering
Б
\end{minipage} & \begin{minipage}[b]{\linewidth}\centering
К
\end{minipage} & \begin{minipage}[b]{\linewidth}\centering
А
\end{minipage} & \begin{minipage}[b]{\linewidth}\centering
Т
\end{minipage} \\
\midrule\noalign{}
\endfirsthead
\toprule\noalign{}
\begin{minipage}[b]{\linewidth}\raggedright
Буква
\end{minipage} & \begin{minipage}[b]{\linewidth}\centering
Ш
\end{minipage} & \begin{minipage}[b]{\linewidth}\centering
О
\end{minipage} & \begin{minipage}[b]{\linewidth}\centering
Р
\end{minipage} & \begin{minipage}[b]{\linewidth}\centering
Х
\end{minipage} & \begin{minipage}[b]{\linewidth}\centering
space
\end{minipage} & \begin{minipage}[b]{\linewidth}\centering
Д
\end{minipage} & \begin{minipage}[b]{\linewidth}\centering
У
\end{minipage} & \begin{minipage}[b]{\linewidth}\centering
Б
\end{minipage} & \begin{minipage}[b]{\linewidth}\centering
К
\end{minipage} & \begin{minipage}[b]{\linewidth}\centering
А
\end{minipage} & \begin{minipage}[b]{\linewidth}\centering
Т
\end{minipage} \\
\midrule\noalign{}
\endhead
\bottomrule\noalign{}
\endlastfoot
Кол-во & 2 & 6 & 2 & 2 & 5 & 2 & 2 & 2 & 3 & 2 & 2 \\
\end{longtable}

\textbf{Всего букв:} 30

\subsection{Метод
Шеннона-Фано}\label{ux43cux435ux442ux43eux434-ux448ux435ux43dux43dux43eux43dux430-ux444ux430ux43dux43e}

\sucsvtable{huffman.csv}{Решение методом Шеннона-Фано}

Итоговый код:

\([\,0111\,\,11\,\,0110\,\,11\,\,0101\,]\allowbreak101\allowbreak[\,11\,\,0000\,]\allowbreak101\allowbreak[\,0100\,\,0011\,\,0010\,\,100\,\,0001\,]\allowbreak101\allowbreak[\,100\,\,0001\,\,100\,]\allowbreak101\allowbreak[\,0010\,\,0011\,\,0100\,\,0000\,\,11\,]\allowbreak101\allowbreak[\,0101\,\,11\,\,0110\,\,11\,\,0111\,]\)

Коэффициент сжатия по формуле 2.1: \(K=120/100=1.2\)

\subsection{Метод
Хаффмана}\label{ux43cux435ux442ux43eux434-ux445ux430ux444ux444ux43cux430ux43dux430}

\image{huffman.png}{Граф для метода Хаффмана}{0.95}

Итоговый код:

\([\,1100\,\,10\,\,0111\,\,10\,\,0110\,]\allowbreak000\allowbreak[\,10\,\,1101\,]\allowbreak000\allowbreak[\,0101\,\,0100\,\,0011\,\,111\,\,0010\,]\allowbreak000\allowbreak[\,111\,\,0010\,\,111\,]\allowbreak000\allowbreak[\,0011\,\,0100\,\,0101\,\,1101\,\,10\,]\allowbreak000\allowbreak[\,0110\,\,10\,\,0111\,\,10\,\,1100\,]\)

Коэффициент сжатия по формуле 2.1: \(K=120/100=1.2\)

\subsection{Арифметическое
кодирование}\label{ux430ux440ux438ux444ux43cux435ux442ux438ux447ux435ux441ux43aux43eux435-ux43aux43eux434ux438ux440ux43eux432ux430ux43dux438ux435}

\begin{longtable}[]{@{}
  >{\centering\arraybackslash}p{(\columnwidth - 20\tabcolsep) * \real{0.0859}}
  >{\centering\arraybackslash}p{(\columnwidth - 20\tabcolsep) * \real{0.0859}}
  >{\centering\arraybackslash}p{(\columnwidth - 20\tabcolsep) * \real{0.0920}}
  >{\centering\arraybackslash}p{(\columnwidth - 20\tabcolsep) * \real{0.0920}}
  >{\centering\arraybackslash}p{(\columnwidth - 20\tabcolsep) * \real{0.0920}}
  >{\centering\arraybackslash}p{(\columnwidth - 20\tabcolsep) * \real{0.0920}}
  >{\centering\arraybackslash}p{(\columnwidth - 20\tabcolsep) * \real{0.0920}}
  >{\centering\arraybackslash}p{(\columnwidth - 20\tabcolsep) * \real{0.0920}}
  >{\centering\arraybackslash}p{(\columnwidth - 20\tabcolsep) * \real{0.0920}}
  >{\centering\arraybackslash}p{(\columnwidth - 20\tabcolsep) * \real{0.0920}}
  >{\centering\arraybackslash}p{(\columnwidth - 20\tabcolsep) * \real{0.0920}}@{}}
\caption{Таблица интервалов}\tabularnewline
\toprule\noalign{}
\begin{minipage}[b]{\linewidth}\centering
Ш
\end{minipage} & \begin{minipage}[b]{\linewidth}\centering
О
\end{minipage} & \begin{minipage}[b]{\linewidth}\centering
Р
\end{minipage} & \begin{minipage}[b]{\linewidth}\centering
Х
\end{minipage} & \begin{minipage}[b]{\linewidth}\centering
space
\end{minipage} & \begin{minipage}[b]{\linewidth}\centering
Д
\end{minipage} & \begin{minipage}[b]{\linewidth}\centering
У
\end{minipage} & \begin{minipage}[b]{\linewidth}\centering
Б
\end{minipage} & \begin{minipage}[b]{\linewidth}\centering
К
\end{minipage} & \begin{minipage}[b]{\linewidth}\centering
А
\end{minipage} & \begin{minipage}[b]{\linewidth}\centering
Т
\end{minipage} \\
\midrule\noalign{}
\endfirsthead
\toprule\noalign{}
\begin{minipage}[b]{\linewidth}\centering
Ш
\end{minipage} & \begin{minipage}[b]{\linewidth}\centering
О
\end{minipage} & \begin{minipage}[b]{\linewidth}\centering
Р
\end{minipage} & \begin{minipage}[b]{\linewidth}\centering
Х
\end{minipage} & \begin{minipage}[b]{\linewidth}\centering
space
\end{minipage} & \begin{minipage}[b]{\linewidth}\centering
Д
\end{minipage} & \begin{minipage}[b]{\linewidth}\centering
У
\end{minipage} & \begin{minipage}[b]{\linewidth}\centering
Б
\end{minipage} & \begin{minipage}[b]{\linewidth}\centering
К
\end{minipage} & \begin{minipage}[b]{\linewidth}\centering
А
\end{minipage} & \begin{minipage}[b]{\linewidth}\centering
Т
\end{minipage} \\
\midrule\noalign{}
\endhead
\bottomrule\noalign{}
\endlastfoot
\(\frac{2}{30}\) & \(\frac{8}{30}\) & \(\frac{10}{30}\) &
\(\frac{12}{30}\) & \(\frac{17}{30}\) & \(\frac{19}{30}\) &
\(\frac{21}{30}\) & \(\frac{23}{30}\) & \(\frac{26}{30}\) &
\(\frac{28}{30}\) & \(\frac{30}{30}\) \\
\(\frac{0}{30}\) & \(\frac{2}{30}\) & \(\frac{8}{30}\) &
\(\frac{10}{30}\) & \(\frac{12}{30}\) & \(\frac{17}{30}\) &
\(\frac{19}{30}\) & \(\frac{21}{30}\) & \(\frac{23}{30}\) &
\(\frac{26}{30}\) & \(\frac{28}{30}\) \\
\end{longtable}

Скрипт на Python представлен в Приложении, результат его работы
изображен на \ref{fig:Результат арифметического кодирования}.

\image{report_images/image-4.png}{Результат арифметического кодирования}{0.95}

Видно, что получившийся полуинтервал имеет начало
\seqsplit{$0.0441034119038795609488903820451530821290744566744192886516829504916878805655507570249379945999303363152647240064389130359306227794852805966939598545145125428666525999348211959113656154842474099719851210466003976762294769287109375$}
и конец
\seqsplit{$0.0441034119038795609488903820451530821290744566744192923055136319621444851745026939416073769704364819281832636103168237518939100236464475654644311480088525252048969141360375214803192579532571671041552008318831212818622589111328125$}.

Исходя из рисунка \ref{fig:Результат арифметического кодирования}, можно
сделать вывод, что сообщение можно закодировать количеством бит равным =
\(171\).

Коэффициент сжатия по формуле 2.1: \(K=248/171=1.45\)

\subsection{Алгоритм
LZW}\label{ux430ux43bux433ux43eux440ux438ux442ux43c-lzw}

Скрипт на Python представлен в Приложении, результат его работы
изображен на \ref{fig:Результат работы LZW}.

\image{report_images/image-3.png}{Результат работы LZW}{0.95}

Коэффициент сжатия по формуле 2.1: \(K=252/240=0.95\)

\section{Вывод}\label{ux432ux44bux432ux43eux434}

В ходе выполнения лабораторной мы сжали исходную строку ``ШОРОХ ОТ ДУБКА
КАК БУДТО ХОРОШ'' 4 разными способами. Для каждого способа мы посчитали
коэффициент сжатия текста, и получили следующие значения:

\begin{enumerate}
\def\labelenumi{\arabic{enumi}.}
\tightlist
\item
  Арифметическое кодирование = \(1.45\)
\item
  Метод Хаффмана = \(1.2\)
\item
  Метод Шенона-Фано = \(1.2\)
\item
  Алгоритм LZW = \(0.95\)
\end{enumerate}

Как мы видим, арифмитическое кодирование имеет самую высокую степень
сжатия, но тем не требует значительно большую мощность вычислительных
ресурсов.

Метод Хаффмана и метод Шенона-Фано имеет одинаковую степень сжатия. Эти
алгоритмы являются простыми в реализации, поэтому для некоторых задач
могут быть весьма эффективными.

Алгоритм LZW имеет степень сжатия меньше единицы. Так получилось, потому
что мы не учитывали то, что для предыдущих алгоритмов нужно передавать
таблицу кодировок. Для алгоритма LZW этого не требуется, что является
ощутимым плюсом.

Полученные навыки пригодятся нам при создании ПО чувствительного к
размеру информации.

\centertitle{Приложение}

\textbf{Листинг арифметического кодирования:}

\lstinputlisting{arithmetic_coding.py}

\textbf{Листинг LZW кодирования:}

\lstinputlisting{lzw_encode.py}

\textbf{Листинг кодирования по словарю:}

\lstinputlisting{fill_word.py}
\end{document}
